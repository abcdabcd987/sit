\section{模块介绍}

\subsection{Color}
\subsubsection{简介}
高亮输出信息,增强可读性。
\subsubsection{实现方法}
\begin{description}
	\item[Windows] 通过Win32 API实现,颜色数量较少,实际效果较差。
	\item[Unix及其他POSIX标准系统] 通过ANSI颜色序列实现,色彩丰富,实际效果较好。
\end{description}

\subsection{Config}
\subsubsection{简介}
sit的配置模块,用来读取和修改作者信息(作者名字及Email地址)。

\subsection{Core}
\subsubsection{简介}
sit的核心模块,sit的所有命令都调用Core中相关函数实现。

\subsection{Diff}
\subsubsection{简介}
用于比较两个文件之间或两个版本之间的差异。
\subsubsection{实现方法}
\begin{description}
	\item[版本之间] 先对两个版本的index进行diff,等到两个版本的文件列表的额差异信息(Add, Rm, Modify, Same),然后对差异信息为Modify的文件调用针对文件的比较算法。
	\item[文件之间] 对两个文件逐行Hash,对得到的Hash值列表执行LCS算法(时间复杂度$O((n+m)*d)$,其中$n$, $m$为两个列表的长度,$d$为输入数据的差异个数)。
\end{description}

\subsection{Hash}
\subsubsection{简介}
Hash在本项目中扮演了非常重要的角色,包括版本的管理和Diff算法的实现。
\subsubsection{Hash在项目中的使用}
使用SHA1算法的Hash值作为文件名来保存文件。虽然效率上会受到一定的影响,但为了保证跨平台的兼容性,SHA1算法直接使用了boost库中的实现而不是git中使用了大量内嵌汇编的SHA1算法。

在针对文件的比较算法中,先使用了MurmurHash3的Hash算法,并且针对32bit和64bit的系统做出了针对性优化。

\subsection{FileSystem}
\subsubsection{简介}
由于Windows和Linux对文件操作的API各不相同,因此使用boost库中的filesystem进行文件操作。同时为了方便代码的编写和实现一些必要的操作,对boost::filesystem进行了再封装。
\subsubsection{主要函数}
\begin{description}
	\item[CompressCopy] 对源文件进行压缩后进行复制,主要用于add命令的实现。
	\item[CompressWrite] 将一个字符串压缩后写入文件,主要用于版本信息的压缩与写入。
	\item[DecompressCopy] 对源文件进行解压后进行复制,主要用于需要从版本库中复制文件的命令的实现。
	\item[DecompressRead] 对一个被压缩的文件使用,返回文件原来的信息,主要用于对版本信息的解压与读取。
	\item[GetRelativePath] 得到路径A相对于路径B的相对路径,用于保证在仓库的子目录中调用sit是可以得到正确的执行结果。
	\item[ListRecursive] 返回包含某一个目录下的所有文件的vector,主要用于实现需要文件夹递归的命令(add, checkout, etc)
\end{description}

\subsection{Index}
\subsubsection{简介}
程序运行时用于处理版本的文件列表信息的数据结构,其本质是对std::unordered\_map的再封装。
\subsubsection{实现细节}
考虑到在程序运行过程中需要处理三种类型的文件列表:
\begin{enumerate}
	\item 保存在".sit/index"中的已暂存的文件信息。
	\item 保存在".sit/objects"中以树形结构储存的文件信息。
	\item 当前工作目录下所有文件的文件信息。
\end{enumerate}
其中所有的文件列表都使用<路径,SHA1值>的键值对保存。

在项目中编写了基类IndexBase,以及其针对文件中的index信息的派生类CommitIndex、针对".sit/index"中的文件信息的派生类Index、针对当前工作目录的所有文件的派生类WorkingIndex。

文件列表的读取和写入通过FileSystem和Objects模块来实现。

\subsection{Objects}
\subsubsection{简介}
作为文件与Index之间的中间数据结构,在Index与Commit文件的转换中具有重要的作用。tree类型的object以树形结构保存了index中的文件列表,清晰地记录了文件和路径的关系。
\subsubsection{实现细节}
包含三种类型的Objects:
\begin{description}
	\item[commit] 该类型的object表示该文件记录的是一个版本信息,包含文件的信息和该commit的信息(作者,时间,以及COMMIT\_MSG)。
	\item[tree] 表示这个文件记录了一个文件列表,该object的内容是同一个文件夹下所有文件的信息,每行一个object,分别代表文件权限,该object的类型,SHA1值,以及object的名字(如果是blob类型则该文件的文件名,如果是tree类型,则表示该文件夹的名字)。
	\item[blob] 一个基本的object类型,表示它是路径树上的叶子节点,是原文件的一个副本。
\end{description}
保存Object的结构体:
\begin{lstlisting}
struct TreeItem {
	int mode;                         // Linux文件权限,如10644
	ObjectType type;                  // 这个Object的类型:tree or blob
	std::string id;                   // 文件的SHA1值,等价于文件的路径
	boost::filesystem::path filename; // 文件路径
};
\end{lstlisting}
保存Commit信息的结构体
\begin{lstlisting}
struct Commit {
	std::string tree;      // 以REPO_ROOT为根的目录树的SHA1 Hash值
	std::string parent;    // 该提交的父版本
	std::string author;    // 作者的名字,Email地址,时间戳,时区
	std::string committer; // 提交者的名字,Email地址,时间戳,时区
	std::string message;   // Commit Messages
};
\end{lstlisting}

\subsection{Refs}
\subsubsection{简介}
用于处理与版本指针相关的引用信息,如,当前工作区的版本(HEAD指针),当前分支的版本指针(指向当前分支的最后一个版本,由于sit目前不支持分支功能,故只有一个默认生成的master分支)。编写该模块的目的主要是为了减少以后需要增加分支功能的时候的工作量。

\subsection{Status}
\subsubsection{简介}
用于处理工作区和index以及版本库之间的关系,输出当前工作区的状态。
\subsubsection{实现细节}
使用Index模块中的WorkingIndex类生成包含当前工作区的所有文件信息的index,会与保存在``.sit/index''中的index以及HEAD指针所指向的commit比较,生成文件的状态(新增,修改,删除及这些改动是否被``.sit/index''所记录)。如果三者完全相同那么认为工作区是clean的。同时,工作区是clean的是checkout命令的对象为commit时能正确执行的前提。

\subsection{Util}
\subsubsection{简介}
重要的一个模块,作为整个工程的工具模块,包含了对boost库中SHA1算法的封装,以及根据当前的objects目录提供的SHA1值的补全功能,增强了sit的易用性。

该模块中定义了sit的异常类SitException,处理由于用户的非法操作造成的错误并给出说明。
